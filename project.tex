% THIS TEMPLATE IS A WORK IN PROGRESS
% Adapted from an original template by faculty at Reykjavik University, Iceland

\documentclass{scrartcl}
\input{File_Setup.tex}
\usepackage{hyperref}


\begin{document}
%Title of the report, name of coworkers and dates (of experiment and of report).
\begin{titlepage}
	\centering
	\includegraphics[width=1\textwidth]{Graphics/MODES.jpeg}\par
	\vspace{2cm}
	%%%% COMMENT OUT irrelevant lines among the 3 below
	{\scshape\LARGE Thermodynamics, \par}  %if you're a CS major
	{\scshape\LARGE Modeling \& Simulations \par}                
	\vspace{1cm}
	{\scshape\Large Winter Semester\par}
    {\scshape\Large 2022/2023\par}
	%{\large \today\par}
	\vfill
	
	%%%% PROJECT TITLE
	{\huge\bfseries Energy Minimization\par}
    {\huge\bfseries of Nanoparticle\par}
	\vfill
	
	%%%% AUTHOR(S)
	{\Large\itshape Changming Chen,}\par
    \vspace{.15cm}
    {\Large\itshape Darren Lam Ming Hui,}\par
    \vspace{.15cm}
    {\Large\itshape Shizhe Xing,}\par
    \vspace{.15cm}
    {\Large\itshape Thanh Mai Vu}\par
	\vspace{.15cm}

	\vfill
	supervised by\par
	%%%% SUPERVISOR(S)
	Abhishek Khetan


	\vfill
% Bottom of the page
\end{titlepage}

\newpage

\begin{acknowledgements}
Before we begin, we would like to acknowledge the work that all our group members have put in. The countless hours and energy spent on this project is paid off tremendously when we finally got the results. We also would like to express our gratitude to other groups for lending a helping hand when we were facing difficulties. Also special thanks to Abhishek Khetan, our supervisor for this project, for his support and inspiration. 
\end{acknowledgements}

    \vspace{1cm}
    
\begin{abstract}
        In this project, we use numerical methods, including line search and conjugate gradient, to study the Minima of Potential Energy using the Lennard-Jones Potential Function and run stochastic simulations with different initial configurations to obtain the putative global potential energy. For reference we take the paper Global Optima of Lennard-Jones Cluster \cite{GOLJC} to compare the results. Several methods and algorithms were explored and implemented to achieve high accuracy. At the end we are able to observe the mechanically stable structure of a nanoparticle by visualizing the cluster at its global optima accurately. 
	    
\end{abstract}
\vspace{1cm}

\begin{keywords}
\centering
        Keywords:\textbf{ Numerical Approach for Potential Energy Minimization with Lennard-Jones Clusters}
\end{keywords}

\newpage



\doublespacing
\tableofcontents
\singlespacing

\newpage

\doublespacing

\section{Introduction}
\par The goal of the project is to find the minimum potential energy so that a nanoparticle conforms to its stable structure. For each simulation, we firstly place the atoms in a three-dimensional cube of size L, which is proportional to the number of atoms N. 
\par For our numerical method, we implement two separate functions, the first one for the total potential energy based on Lennard-Jones equation, which will then be minimized and the second one is its derivative, the force function, what is needed to perform minimization.
\par The next step is to implement the line search function, which will find the first minimum along a search direction. Once we reach the minimum area, we apply the Golden Section Method to locate the minimum. The reason is to take advantages of smart algorithm, reduce the effort of function calculation and thus computational cost.
\par Finally, we apply conjugate gradient descent which outperforms steepest descent by taking previous steps, eliminating the oscillatory behaviour and gives us an efficient algorithm. It will calculate different search directions, call line search and return the value, which is a nearby local minimum with a given initial value, a nearest stable conformation.
\par We then run many simulations to have several local minima with different energy values resulting from different initial configurations. The one that has the lowest energy is selected, a ‘greedy’ step is taken, we get our putative global minimum.
\par We have opted for 3 different versions of codes for the solution for this assignment which will be introduced later in the \hyperref[sec:Solution]{solution section}. Each with slight changes to the other. 
\par 

\newpage
\section{Related Work}
\par As we go to the higher number of atoms N, the number of local minima will grow exponentially. Therefore, we can easily end up in one of the many local minima while conducting the conjugate gradient. To validate that our algorithm could reach the lowest possible minima after some simulations, we constantly compare our results with the table shown below. 

\begin{table} [h]
    \centering
    \label{tab:table1}
 
    \begin{tabular}{|c|c|c|c|}
    \hline
   N      &  -U       &   N    &  -U \\
   \hline
    2     &  1.000000  &   14  &  47.845157\\
    3     &  3.000000  &   15  &  52.322627\\
    4     &  6.000000  &   16  &  56.815742\\
    5     &  9.1003852 &   17  &  61.317995\\
    6     &  12.712062 &   18  &  66.530949\\
    7     &  16.505384 &   19  &  72.659782\\
    8     &  19.821489 &   20  &  77.177043\\
    9     &  24.113360 &   21  &  81.684571\\
    10    &  28.422532 &   22  &  86.809782\\
    11    &  32.765970 &   23  &  92.844472\\
    12    &  37.967600 &   24  &  97.348815\\
    13    &  44.326801 &   25  &  102.372663\\      
    \hline           
    \end{tabular}
    \caption{Values given by Global Optima of Lennard-Jones Clusters} \cite{GOLJC}
\end{table}

\par During the simulations some significant deviation from the values stated here are observed which will be shown and discussed here \hyperref[sec:Results and Discussions]{Results and Discussions}

\newpage
\section{Solution}
\label{sec:Solution}
\subsection{Energy Function}
\par The total potential energy 
\begin{equation}
    U^{*} = \sum_{i} \alpha|r_{i}|^{2} + \sum_{i<j}u(r_{ij}) \quad , \quad
    \alpha = 0.0001N^{-2/3}
\end{equation}
    with $u(r_{ij})$ given by

\begin{equation}
    u(r_{ij}) = \left\{
                \begin{array}{ll}
                    4\epsilon[(r_{ij}/\sigma)^{-12} - (r_{ij}/\sigma)^{-6}] - 4\epsilon[(r_{c}/\sigma)^{-12} - (r_{c}/\sigma)^{-6}] &  r_{c} \geq r_{ij} \\
                    0 & r_{c} > r_{ij}  
                \end{array}
                r_{c} = 2.5\sigma
\right.
\end{equation}
To speed up the code we have
\begin{equation}
    term = 4\epsilon((r_{c}/\sigma)^{-12} - (r_{c}/\sigma)^{-6}) \\
\end{equation}
\begin{equation}
    term1 = (\sigma/r_{ij})^{6}
\end{equation}
so that we end up with the simplified energy function as
\begin{equation}
    U^{*} = 4\epsilon term1(term1-1)-term
\end{equation}

\subsection{Force Function}
\par The force function is the derivative of the energy function with 
\begin{equation}
    u_{ij}^{'} = -4\epsilon[12 (\sigma^{12}/r_{ij}) - 6 (\sigma^{6}/r_{ij}^{7})]
\end{equation}
For simplification we write 
\begin{equation}
    term = (\sigma/r_{ij})^{6}
\end{equation}
and the final implemented force equation is given by
\begin{equation}
    f_{ij} = -2\alpha r_{i}  -  \sum_{i\ne j} 24\epsilon * \hat{r_{ij}}/r_{ij}^{2} *(term(2term-1)) \quad, \hat{r_{ij}} = r_{j} -r_{i}
\end{equation}


\subsection{Line Search} 
\cite{alam_hinchchliffe_molecular_2003}
\textbf{Algorithm: } \\
\textbf{Step 1: } Start with an initial set of coordinates $r_{0}^{N} $and a search direction \textit{d}\\
\textbf{Step 2: } 
Assign $r_{1}^{N} = r_{0}^{N} + \delta d$ and $r_{2}^{N} = r_{1}^{N} + ,\delta d$ \\
\textbf{Step 3: } 
Check if $U(r_{2}^{N}) > U(r_{1}^{N})$ and $U(r_{0}^{N}) > U(r_{1}^{N})$ \\
-> If \textbf{No}, we have not reached the minimum region, then \\
\textbf{Step 4a: } Shift values as $r_{0}^{N} = r_{1}^{N}, r_{1}^{N} = r_{2}^{N}, r_{2}^{N} = r_{1}^{N} + \delta d$, and \textbf{go back to step 3} \\
-> If \textbf{Yes} then \\
\textbf{Step 4b: } Locate the minimum $r_{min}^{N}$ between $r_{0}^{N}$, $r_{1}^{N}$ and $r_{2}^{N}$ by Golden Section method. \\
\textbf{Step 5: } \\
Calculate a new search direction along $r_{min}^{N}$ and \textbf{go back to step 2}. \\
Repeat the procedure until the change in energy $\Delta$U is less than \textit{LSstol} \ref{eq:7} \\
\textbf{Step size (learning rate): } \\
\textbf{1. }Choose a reasonably small step size $\delta=0.01$ \\
\textbf{2. }Normalize the given search section row-wise with Euclidean norm \\
We do not normalize the whole matrix by dividing all of its elements by one value, because when some atoms build a cluster, they then have a huge force to each other compared to the one towards those which are far away. Atoms in different clusters do not feel forced by other clusters which makes them only move around themselves and different clusters do not come together. They than build many clusters and thus the energy is not minimized, which is not supposed to be. \\
The \textbf{reason} for row-wise norm is to allow all atoms move with the same peace at the same time and not any is left behind. All atoms come than together to the origin of the coordinate. We can then get the next lowest possible energy and continue searching. \\
\textbf{3. }Our method is conjugate gradient descent. With some mathematical background descent direction means direction that leads to lower function values. That's why $U_{1}$ should always be lower than $U_{0}$, if it's not the case, we have to reduce $\delta$ by half each time and keep trying until we get the optimal $\delta$.

\newpage
\subsubsection{Golden Ratio Method}
\par The symbol $\varphi$ represents the Golden Ratio with the equation of 
\begin{equation} \label{eq:4}
    \varphi = 1/2(1 - \sqrt{5}) \quad \textrm{or} \quad \varphi = 1/2(1 +\sqrt{5})
\end{equation}
\par Assuming we have two points $x_{l}$ and $x_{u}$ (lower and upper boundaries), which we have obtained from the line search. We can calculate the step taken with the golden ratio.
\begin{equation} \label{eq:5}
    h = (x_{u} - x_{l}) \varphi
\end{equation}
We then calculate 
\begin{equation} \label{eq:6}
    x_{1} = x_{u} + h \quad \textrm{and} \quad x_{2} = x_{l} - h
\end{equation}
At each iteration we shift the points $x_{l}$ and $x_{u}$ close to the minimum energy until a certain tolerance is reached, here it is stated as LStol = $10^{-8}$. With the condition
\begin{equation} \label{eq:7}
    |U_{u} - U_{l}| <= LStol|U_{0}| 
\end{equation}

\begin{figure} [h!]
    \centering
    \includegraphics[width=0.8\textwidth]{Graphics/golden_section_search_iteration.png}
    \caption{Golden Ratio Method for finding the maximum (an analogue for minimum)}
    \label{fig:my_label 1}
\end{figure}

\newpage
\subsection{Conjugate Gradient}
\par We then need a descent direction for our line search, which brings our function fastest and most efficiently to the next local minimum. With steepest descent descent direction, it usually leads to a zig-zag situation, where function value just changes within a very small amount range. To address the mentioned problem, there is conjugate gradient descent which has an ability able to rectify the right angle steps that the line search then takes. \\
The direction is calculated using the formula
\begin{equation} \label{eq:2}
    d_{i}^{N} = f_{i}^{N} + \gamma_{i}d_{i-1}^{N}
\end{equation} \label{eq:3}
which takes into account the previous step.\\
We calculate the $\gamma$ using the ratio discovered by Polak–Ribière \cite{Polak–Ribière}
\begin{equation}
    \gamma_{i}^{N} = \frac{(f_{i}^{N} - f_{i-1}^{N})f_{i}^{N}}{f_{i-1}^{N}f_{i-1}^{N}}\
\end{equation}
\par We take only $\gamma$, which is greater than 0, into account, because the previous step leading to minimum along a searching line should not be undone or removed.

\newpage
\section{Results and Discussions}
\label{sec:Results and Discussions}
\par Before starting with the results and discussions, we have to mention that there are three varying versions of code. Only the third version is the one that we use as our final solution. And below we will also justify the differences between them.
\subsection{Version 1}
\begin{table} [h] \label{tab:table2}
    \centering
 
    \begin{tabular}{|c|c|c|c|c|c|}
    \hline
   N      &  -U        &  Deviation $\%$ &  N   &  -U   &  Deviation $\%$\\
   \hline
    2     &  0,983683  &  1,6316891 &14  &  46,377772 & 3,0669456\\
    3     &  2,951049  &  1,6316893 &15  &  50,643140 & 3,2098682\\
    4     &  5,902099  &  1,6316895 &16  &  54,007955 & 4,9419171\\
    5     &  8,940683  &  1,7548923 &17  &  58,240797 & 5,0184256\\
    6     &  12,058173 &  5,1438459 &18  &  63,396732 & 4,7109156\\
    7     &  16,162725 &  2,0760438 &19  &  67,162387 & 7,5659390\\
    8     &  19,364613 &  2,3049528 &20  &  70,738328 & 8,3427849\\
    9     &  23,525946 &  2,4360505 &21  &  73,055061 & 10,5644306\\
    10    &  27,688260 &  2,5834160 &22  &  80,029805 & 7,8101528\\
    11    &  31,868531 &  2,7389363 &23  &  85,061427 & 8,3828852\\
    12    &  35,235303 &  7,1963900 &24  &  87,303688 & 10,3186952\\
    13    &  43,054067 &  2,8712524 &25  &  88,698210 & 13,3575241\\      \hline           
    \end{tabular}
    \caption{Acquired values from the 1st version of simulation with the deviation from Table 1~\ref{tab:table1}}
\end{table}
\subsubsection{Discussion}
\par In the first version, all the implemented algorithm was written as instructed in the exercises. However, the results wasn't as accurate as it is supposed to be, which can be obtained from the deviation in the table above. The deviation grows exponentially with greater number of atoms. Therefore, another method was tried out to with some optimization processes and the second version was produced.

\newpage
\subsection{Version 2}

\begin{table} [h] \label{tab:table3}
    \centering
 
    \begin{tabular}{|c|c|c|c|c|c|}
    \hline
   N      &  -U        &  Deviation $\%$ &  N   &  -U   &  Deviation $\%$\\
   \hline
    2     &  1,000000  &  0,0000000  &14  &  47,845143 & 0,0000291\\
    3     &  3,000000  &  0,0000000  &15  &  52,322594 & 0,0000639\\
    4     &  6,000000  &  0,0000002  &16  &  55,905160 & 1,6026927\\
    5     &  9,103852  &  0,0380971  &17  &  61,306964 & 0,0179892\\
    6     &  12,712058 &  0,0000335  &18  &  66,284446 & 0,3705091\\
    7     &  16,505383 &  0,0000099  &19  &  72,658218 & 0,0021520\\
    8     &  19,821485 &  0,0000180  &20  &  77,177000 & 0,0000563\\
    9     &  24,113353 &  0,0000287  &21  &  81,650734 & 0,0414243\\
    10    &  28,422517 &  0,0000536  &22  &  86,179764 & 0,7257449\\
    11    &  32,765955 &  0,0000473  &23  &  90,732832 & 2,2743841\\
    12    &  37,967588 &  0,0000316  &24  &  94,500554 & 2,9258306\\
    13    &  44,326771 &  0,0000677  &25  &  101,026652 & 1,3148149\\      
    \hline           
    \end{tabular}
    \caption{Acquired values from the 2nd version of simulation with the deviation from Table 1~\ref{tab:table1}}
\end{table}

\subsubsection{Discussion}
\par The idea for the second version is that the observed deviation was due to the constraints
\begin{equation} \label{eq:8}
    r_{min} = r_{ij} - Lround(r_{ij}/L) \quad and \quad r_{ij} <r_{c}
\end{equation} 
The $r_{min}$ term allows atoms in different boxes in the lattice to interact with each other. It then often leads to the case when clusters are in the boundaries between boxes. When the density is small enough, at their global minimal potential energy, atoms form a cluster in the middle of the box, the distance between two atoms is always less then L/2, which results in interactions just within one box and thus the equation is no more needed. As the result, we eliminated it.
\par When there are many atoms beforehand, the distance between them are large. Atoms in the neighborhood immediately start conforming their cluster which is isolated from others. The distance between clusters is sometimes greater than ${r_{c}}$, where atoms do not feel force from others clusters anymore and stay still far away from each other. This phenomenon causes the simulation to reach a higher local minima. As a solution, the constraint is eliminated.
\par It does show a better result compared to the previous version, but it still deviates at higher N due to the increase of number of local minima. We couldn't reach the lowest of them, many simulation runs are needed to randomly get the better initial guess, which needs a longer runtime.
\newpage
\subsection{Version 3 (Our Final Solution)}

\begin{table} [h] \label{tab:table4}
    \centering
 
    \begin{tabular}{|c|c|c|c|c|c|}
    \hline
   N      &  -U        &  Deviation $\%$ &  N   &  -U   &  Deviation $\%$\\
   \hline
    2     &  1,000000  &  0,0000000  &14  &  47,843645 & 0,0031608\\
    3     &  3,000000  &  0,0000003  &15  &  52,321065 & 0,0029854\\
    4     &  5,999995  &  0,0000758  &16  &  56,814379 & 0,0023983\\
    5     &  9,103834  &  0,0378977  &17  &  61,315669 & 0,0037937\\
    6     &  12,712035 &  0,0002089  &18  &  66,281077 & 0,3755720\\
    7     &  16,505221 &  0,0009904  &19  &  72,656204 & 0,0049245\\
    8     &  19,821288 &  0,0010156  &20  &  77,170505 & 0,0084717\\
    9     &  24,112981 &  0,0015735  &21  &  81,674564 & 0,0122508\\
    10    &  28,422088 &  0,0015638  &22  &  86,788240 & 0,0248156\\
    11    &  32,765364 &  0,0018505  &23  &  92,835814 & 0,0093250\\
    12    &  37,966652 &  0,0024980  &24  &  97,327194 & 0,0222096\\
    13    &  44,326277 &  0,0011825  &25  &  102,352337 & 0,0198548\\      \hline           
    \end{tabular}
    \caption{Acquired values from the 3nd version of simulation with the deviation from Table 1~\ref{tab:table1}}
\end{table}

\subsubsection{Discussion}
\par Instead of randomly get a new initial guess every time we reach one minimum with conjugate gradient, we decided not to stop but just go one more step further to the next local minimum, which can be reasonably larger. From there, we have a good initial guess, what helps us to have a new better set of search directions. We know the path that we have gone though and now we take wiser steps back to the minimum and then to the lower and lower energy value. \par
From our observation, steepest descent gives a moderate step size and a good decrease in function value, when conjugate gradient take too large steps and often end up somewhere too far away from the actual putative global minimum, so we use force as search direction. Moreover, we do not have tolerance value as stop condition anymore, we allow our function to jump from one local minimum to another and stop the algorithm after some reasonable number of steps.
\par With that modification we are now able to have a version that is fast (finished within ten hours compared to two days as previous) with small to zero deviation. \\

\newpage
\section{Graphs} \label{sec:Graphs}
    \subsection{Task 2}
    
    \begin{figure}[h!]
        \centering
        \includegraphics[width=0.8\textwidth]{Graphics/U_min.png}
        \caption{Minimum energy progression at each N}
        \label{fig:my_label 2}
    \end{figure}

    \begin{figure}[h!]
        \centering
        \includegraphics[width=0.8\textwidth]{Graphics/CGSteps.png}
        \caption{Conjugate gradient step progression at each N}
        \label{fig:my_label 3}
    \end{figure}

\newpage
    \subsection{Task 3}

    \begin{figure}[h!]
        \centering
        \includegraphics[width=0.9\textwidth]{Graphics/Task3_graphs_2to9 - Page 1.png}
        \caption{Evolution of the minimum energy at each N (2-9)}
        \label{fig:my_label 4}
    \end{figure}
    
\newpage

    \begin{figure}[h!]
        \centering
        \includegraphics[width=0.9\textwidth]{Graphics/Task3_graphs_10to17 - Page 3.png}
        \caption{Evolution of the minimum energy at each N (10-17)}
        \label{fig:my_label 4}
    \end{figure}

\newpage

    \begin{figure}[h!]
        \centering
        \includegraphics[width=0.9\textwidth]{Graphics/Task3_graphs_18to25 - Page 4.png}
        \caption{Evolution of the minimum energy at each N (18-25)}
        \label{fig:my_label 4}
    \end{figure}

    \begin{figure}[h!]
        \centering
        \includegraphics[width=0.9\textwidth]{Graphics/Task3_vis_2 - Page 2.png}
        \caption{Visualization of each N}
        \label{fig:my_label 5}
    \end{figure}

\newpage
    \subsection{Task 4}
    The macroscopic scaling estimate for nanoparticle energy
    \begin{equation}
        U_{macro}(N) = a + bN^{2/3} +cN
    \end{equation}

    \begin{table}[h!]
        \centering
        
        \begin{tabular}{|c|c|c|}
        \hline
           a  &  b  &  c\\
        \hline 
           -3.078349   & 11.453650 & -7.907704\\
        \hline
        \end{tabular}
        
        \caption{The values of a, b and c}
        \label{tab:table5}
    \end{table}
    
    \begin{figure}[h!]
        \centering
        \includegraphics[width=0.8\textwidth]{Graphics/estimate.png}
        \caption{Fitting the minimum potential energy data to the given formula}
        \label{fig:my_label 6}
    \end{figure}
    
\newpage

\subsection{Bonus Task}
\par Results for $N/V=0.01$
\subsubsection{Task 2}

    \begin{figure}[h!]
        \centering
        \includegraphics[width=0.7\textwidth]{Graphics/U_min_vs_N.png}
        \caption{Minimum energy progression at each N}
        \label{fig:my_label 6}
    \end{figure}

    \begin{table} [h]
    \centering
    \label{tab:table1}
 
    \begin{tabular}{|c|c|c|c|}
    \hline
   N      &  -U       &   N    &  -U \\
   \hline
    2     &  1.000000  &   14  &  47.843584\\
    3     &  3.000000  &   15  &  52.320602\\
    4     &  6.000000  &   16  &  56.813694\\
    5     &  9.103835 &   17  &  61.315664\\
    6     &  12.302552 &   18  &  66.526096\\
    7     &  16.505135 &   19  &  72.656046\\
    8     &  19.821304 &   20  &  77.172222\\
    9     &  24.113126 &   21  &  81.677603\\
    10    &  28.421965 &   22  &  86.800246\\
    11    &  32.765046 &   23  &  91.386532\\
    12    &  37.966161 &   24  &  97.335513\\
    13    &  44.325964 &   25  &  101.853382\\      
    \hline           
    \end{tabular}
    \caption{Valeus for the bonus task with $N/V=0.01$} \cite{GOLJC}
\end{table}

\newpage

\subsubsection{Task 4}

    \begin{table}[h!]
    \centering
    \begin{tabular}{|c|c|c|}
        \hline
        
           a  &  b  &  c\\
        \hline 
           -2.49912   & 11.012015 & -7.765069\\
        \hline
        \end{tabular}
    \end{table}
        
    \begin{figure}[h!]
        \centering
        \includegraphics[width=0.6\textwidth]{Graphics/estimate_bonus.png}
        \caption{Fitting the minimum potential energy data to the given formula}
        \label{fig:my_label 6}
    \end{figure}

\newpage

\section{Discussion}

\par From the graph \ref{fig:my_label 2} it is observed that the minimum energy decreases as the number of atoms increases. Due to the increasing number of atoms and their bonds, more energy is to be released for them to come close together, for a perfectly stable conformation of a nanoparticle. 
\par As the number of atoms increases, more conjugate gradient steps are to be performed, because the structure is complicated at first, atoms need to move more to find their most minimum potential energy state. This is observed in graph \ref{fig:my_label 3}.
\par The minimum energy decreases with the number of conjugate gradient steps\ref{fig:my_label 4}. However, at N=2 there wasn't a graph line observed. That is because the the minimum state is already achieved after the first conjugate gradient step. 
\par Now let's discuss one interesting conformation in this assignment. In the article \cite{GOLJC}, there was mention of the 'magic number'. In this assignment the visualization of the conformation at the 'magic number' is done with the python extension "ase"

\begin{figure}[h!]
    \centering
    \subfigure{\includegraphics[width=0.3\textwidth]{Graphics/n13x.png}} 
    \subfigure{\includegraphics[width=0.3\textwidth]{Graphics/n13y.png}} 
    \subfigure{\includegraphics[width=0.3\textwidth]{Graphics/n13z.png}}
    \caption{Visualisation of the 'magic number' N=13}
    \label{fig:my_label7}
\end{figure}
Shown here is the single shell of Mackay's icosahedral conformation of N=13. Single shell means there one layer of atoms wrapping around a single core atom. There is also other 'magic number' at N=55 and N=147 with each correspondingly has two and three shell of Mackay's icosahedra.


\newpage
\section{Conclusion}
\par From the graph section \ref{sec:Graphs} the minimum potential energy is reached and the atoms do conform into a nanoparticle at their stable configuration. All the algorithms mentioned was applied successfully. Although, at the initial stage the results were not as promising. That has lead to some more numerical optimization of our own so that the algorithm runs more efficiently. 
\par The scalability of the code is still limited and the algorithm may need further modifications to reach the best accuracy when there are changes in given parameters. For the smaller density like $N/V=0.01$, $\alpha$ should be increased to $0.01N^{-2/3}$ so that atoms near the boundaries move to the origin. For the larger number of atoms higher than 25 like N=55, there was still a deviation of nine units. The Lennard-Jones cluster is however shown up in their 'magically' shape.

\par Below is the conformation at N=55, the less uniform structure (which contradicts what is mentioned here \cite{GOLJC}) shows that the algorithm is less scalable at higher N, but still good enough to form a magic cluster.

\begin{figure}[h!]
        \centering
        \includegraphics[width=0.3\textwidth]{N=55.jpg}
        \caption{Conformation at N=55}
        \label{fig:my_label 6}
\end{figure}

The whole assignment could take a whole new approach if we also added velocities to the atoms, bonded interactions (intramolecular) and also other non-bonded interactions (intermolecular). But we have achieved the first milestone and solved our basic problem - potential energy minimization for  microclusters of N neutral atoms interacting pairwise via the 
Lennard-Jones potential functions.

\newpage
\singlespacing
\bibliographystyle{IEEEtran}
\bibliography{references}


%------ To create Appendix with additional stuff -------%
%\newpage
%\appendix
%\section{Appendix}
%Put data files, CAD drawings, additional sketches, etc.

\end{document}
